\documentclass{frontiersSCNS}

\usepackage{url,lineno}
\linenumbers

\def\keyFont{\fontsize{8}{11}\helveticabold }
\def\firstAuthorLast{Bekolay {et~al.}}
\def\Authors{Trevor Bekolay\,$^{1,*}$,
  Terrence C. Stewart\,$^{1}$,
  Eric Hunsberger\,$^1$,
  Andrew Mundy\,$^2$ and
  Chris Eliasmith\,$^1$}
\def\Address{$^{1}$Centre for Theoretical Neuroscience,
  University of Waterloo,
  Waterloo, ON, Canada\\
  $^{2}$ Advanced Processors Technologies Research Group,
  School of Computer Science,
  University of Manchester,
  Manchester, UK}
\def\corrAuthor{Trevor Bekolay}
\def\corrAddress{Centre for Theoretical Neuroscience,
  David R. Cheriton School of Computer Science,
  University of Waterloo,
  200 University Avenue West,
  Waterloo, ON  N2L~3G1, Canada}
\def\corrEmail{tbekolay@uwaterloo.ca}

\begin{document}
\onecolumn
\firstpage{1}

\title[Benchmarking neuromorphic systems with Nengo]{Benchmarking neuromorphic systems with Nengo}
\author[\firstAuthorLast ]{\Authors}
\address{}
\correspondance{}
\extraAuth{}
\topic{Benchmarks and Challenges for Neuromorphic Engineering}

\maketitle

\begin{abstract}

  Abstract

  \tiny
  % 5--8 keywords
  \keyFont{ \section{Keywords:}  }
\end{abstract}

\section{Introduction}

Benchmarking is a notoriously difficult task...

Nengo is a rigorously tested software package
for building and simulating
large-scale neural models
that can perform cognitively relevant tasks.
It provides a high-level API (frontend)
that can express large-scale models concisely
and in a platform-independent manner.
Several Nengo-compatible simulators (backends)
have been developed that can
run Nengo models on diverse platforms,
including neuromorphic hardware.
Nengo's separation between frontend and backend
and rigorous test suite
provide standardized comparisons
between different neuromorphic implementations
that are geared toward (???) functional performance,
rather than artificial metrics
that may or may not contribute to functional performance.

Other software that do similar things:
many neural simulators exist, and have tests,
but don't have the high-level API
that Nengo has.
On the other hand, some simulators
have high-level APIs
(Topographica?) but a) aren't functional
and b) can't use diverse backends.

??? In the subsequent sections,
we detail the architecture of Nengo's
frontend and backend,
and describe what each backend
is required to implement.
We explain Nengo's testing framework,
including descriptions of the testing fixtures
used to collect and visualize benchmarks.
We then list the metrics that will be collected,
and the backends on which we will collect metrics.
Finally, we will show and discuss
the results of collecting those metrics
for each backend.

\section{Background}

There are two key features of Nengo 2.0
that enable rapid benchmarking of neuromorphic systems.
The first is a decoupling of model creation and simulation,
resulting in a platform independent frontend interface
to any backend that implements a certain set of requirements.
The second is a ``fuzzy'' test suite ensuring that
Nengo can be used to create models that solve cognitive tasks.
This test suite makes ample use of test fixtures
to enable data collection while testing on arbitrary backends.

\subsection{Nengo architecture}

Nengo has a strict separation between
the frontend and backend.
The frontend exposes a modelling interface
that uses Python and NumPy \cite{???}
to define models concisely.
Backends are responsible for transforming
those frontend objects into code that
can be run on the target platform.
While each backend must be exposed
through Python, this requirement
does not significantly limit
flexibility in the backend.
A backend can be implemented in C and exposed
through Python bindings,
or can be run as a separate process
managed by the Python backend,
with data transmitted to Python through sockets
or other inter-process communication protocols.

\subsubsection{Frontend interface}

Go over the nengo objects

Unlike other programming models,
it's not like... for sure.
Go over symbolic dists.

Some things are suggestions.
Go over backend hints.

\subsubsection{Backend requirements}

Go over the Simulator object, and required methods.

Go over (briefly) translation of frontend to backend.
Needed: some kind of neuron model; takes current,
produces output (rates, spikes, whatever).
Needed: Ops (look at all ops)

\subsection{Fuzzy testing}

As of version 2.0.1, Nengo's test suite contains 618 tests.
Some test are unit tests though; how many?
In all, ??? can be considered ``functional'' tests,
as they construct a model,
build a simulator object (see previous sec???)
and test that the output of that model
matches the desired output,
to a certain tolerance.

Unlike traditional software testing,
there can be significant variability
in many aspects of a Nengo model.
Many model parameters, for example,
can be randomly generated;
other aspects of a model,
such as injected noise,
are necessarily random.
Noise is a fundamental property
of neuromorphic systems.
The accuracy of any large-scale model
is dependent on many factors,
including the number of neurons used
to implement a particular task.
For this reason,
functional tests can only
ensure that the backend implements
the system described by the frontend
well enough.
Each test must determine what
``well enough'' means for that particular network.
While this introduces some subjectivity
to these benchmarks,
it is nearly impossible to create
completely objective benchmarks.
??? more or remove last bit

Nengo's test suite employs the \texttt{pytest}
testing framework \cite{???}.
??? introduce test fixtures

\subsubsection{Test fixtures}

\paragraph{\texttt{rng} \& \texttt{seed}} to manage random factors

\paragraph{\texttt{Simulator}} to switch backend

\paragraph{\texttt{nl}} to switch neuron type

\paragraph{\texttt{plt}} to save plots

\paragraph{\texttt{analytics}} to save data

\paragraph{\texttt{logger}} to save text

\section{Methods}

\subsection{Metrics collected}

\subsubsection{Compliance}

\subsubsection{Accuracy}

\subsubsection{Speed}

\subsection{Backends tested}

\paragraph{Reference}

\paragraph{Simplified reference}

\paragraph{OpenCL}

\paragraph{SpiNNaker hardware}

\paragraph{Neurogrid simulation}

\section{Results}

\section{Discussion}

\subsection{Data Sharing}

%% Frontiers supports the policy of data sharing, and authors are advised
%% to make freely available any materials and information described in
%% their article, and any data relevant to the article (while not
%% compromising confidentiality in the context of human-subject research)
%% that may be reasonably requested by others for the purpose of academic
%% and non-commercial research. In regards to deposition of data and data
%% sharing through databases, Frontiers urges authors to comply with the
%% current best practices within their discipline.

\section*{Disclosure/Conflict-of-Interest Statement}

%Frontiers follows the recommendations by the International Committee
%of Medical Journal Editors
%(http://www.icmje.org/ethical_4conflicts.html) which require that all
%financial, commercial or other relationships that might be perceived
%by the academic community as representing a potential conflict of
%interest must be disclosed. If no such relationship exists, authors
%will be asked to declare that the research was conducted in the
%absence of any commercial or financial relationships that could be
%construed as a potential conflict of interest. When disclosing the
%potential conflict of interest, the authors need to address the
%following points: • Did you or your institution at any time receive
%payment or services from a third party for any aspect of the
%submitted work?  • Please declare financial relationships with
%entities that could be perceived to influence, or that give the
%appearance of potentially influencing, what you wrote in the
%submitted work.  • Please declare patents and copyrights, whether
%pending, issued, licensed and/or receiving royalties relevant to the
%work.  • Please state other relationships or activities that readers
%could perceive to have influenced, or that give the appearance of
%potentially influencing, what you wrote in the submitted work.

%% The authors declare that the research was conducted in the absence of
%% any commercial or financial relationships that could be construed as a
%% potential conflict of interest.

\section*{Author Contributions}

%When determining authorship the following criteria should be observed:
%• Substantial contributions to the conception or design of the work; or the acquisition, analysis, or interpretation of data for the work; AND
%• Drafting the work or revising it critically for important intellectual content; AND
%• Final approval of the version to be published ; AND
%• Agreement to be accountable for all aspects of the work in ensuring that questions related to the accuracy or integrity of any part of the work are appropriately investigated and resolved.
%Contributors who meet fewer than all 4 of the above criteria for authorship should not be listed as authors, but they should be acknowledged. (http://www.icmje.org/roles_a.html)

%% The statement about the authors and contributors can be up to several
%% sentences long, describing the tasks of individual authors referred to
%% by their initials and should be included at the end of the manuscript
%% before the References section.

\section*{Acknowledgments}
%% Acknowledgments

\paragraph{Funding\textcolon} %% Funding

\section*{Supplemental Data}
%% Supplementary Material should be uploaded separately on submission, if
%% there are Supplementary Figures, please include the caption in the
%% same file as the figure. LaTeX Supplementary Material templates can be
%% found in the Frontiers LaTeX folder

\bibliographystyle{frontiersinSCNS}
\bibliography{paper}

\section*{Figures}

\end{document}


%% \begin{table}[!t]
%% \textbf{\refstepcounter{table}\label{Tab:01} Table \arabic{table}.}{
%%   Maximum size of the Manuscript }

%% \processtable{ }
%% {\begin{tabular}{lllll}\toprule
%%  & Abstract max. legth (incl. spaces) & Figures or tables & Manuscript max. length & Final PDF length\\\midrule
%%  Clinical Case Study & & & &\\
%%  Clinical Trial & & & &\\
%%  Hypothesis and Theory & & & &\\
%%  Methods & 2000 characters  & 15 & 12000 words & 12 pages\\
%%  Original Research & & & &\\
%%  Review & & & &\\
%%  Technology Report & & & &\\
%%  Focused Review & 2000 characters & 5 & 5000 words & 5 pages\\
%%  CPC &  1250 characters& 6 & 2500 words & 4 pages\\
%%  Perspective & 1250 characters & 2 & 3000 words & 3 pages\\
%%  Mini Review & & & &\\
%%  Classification & 1250 characters & 10 & 2000 words & 12 pages\\
%%  Editorial & none & none & 1000 words & 1 page \\
%%  Book review & & & &\\
%%  Frontiers Commentary & none & 1 & 1000 words & 1 page\\
%%  General Commentary & & & &\\
%%  Field Grand Challenge & & & &\\
%%  Opinion & none & 1 & 2000 words & 2 pages\\
%%  Specialty Grand Challenge& & & &\\\botrule
%% \end{tabular}}{}
%% \end{table}

%% \begin{figure}[h!]
%% \begin{center}
%% \includegraphics[width=10cm]{logo1}
%% \end{center}
%% \textbf{\refstepcounter{figure}\label{fig:01} Figure \arabic{figure}.}
%%        { Enter the caption for your figure here.  Repeat as necessary
%%          for each of your figures }
%% \end{figure}
