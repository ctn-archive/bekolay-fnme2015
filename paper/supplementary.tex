%%% You will need to have the following packages installed: datetime, fmtcount, etoolbox, fcprefix, which are normally inlcuded in WinEdt. %%%
\documentclass{frontiers_suppmat} % for all articles

\usepackage{url}

\graphicspath{{../figures/}}

\def\articleType{Supplementary Material}
\def\keyFont{\fontsize{8}{11}\helveticabold }
\def\firstAuthorLast{Bekolay {et~al.}}
\def\Authors{Trevor Bekolay\,$^{1,*}$,
  Terrence C. Stewart\,$^{1}$,
  Eric Hunsberger\,$^1$,
  Andrew Mundy\,$^2$ and
  Chris Eliasmith\,$^1$}
\def\Address{$^{1}$Centre for Theoretical Neuroscience,
  University of Waterloo,
  Waterloo, ON, Canada\\
  $^{2}$ Advanced Processors Technologies Research Group,
  School of Computer Science,
  University of Manchester,
  Manchester, UK}
\def\corrAuthor{Trevor Bekolay}
\def\corrAddress{Centre for Theoretical Neuroscience,
  David R. Cheriton School of Computer Science,
  University of Waterloo,
  200 University Avenue West,
  Waterloo, ON  N2L~3G1, Canada}
\def\corrEmail{tbekolay@uwaterloo.ca}

% \color{FrontiersColor} Is the color used in the Journal name, in the title, and the names of the sections.

\begin{document}
\onecolumn
\firstpage{1}

\title[Supplementary Material]{{\helveticaitalic{Supplementary Material}}:\\ \helvetica{ Benchmarking neuromorphic systems with Nengo}}
\author[\firstAuthorLast ]{\Authors}
\address{}
\correspondance{}
\extraAuth{}
\topic{Benchmarks and Challenges for Neuromorphic Engineering}

\maketitle

\section{Supplementary Tables and Figures}

\begin{table}[!ht]
\textbf{\refstepcounter{table}\label{Tab:01} Supplementary Table \arabic{table}.}{
  Software versions used for benchmarking.
  The PyPI name is the unique identifier used for installing the Python package
  through the Python Package Index (PyPI). Packages can be found by
  visiting \url{https://pypi.python.org/pypi/<PyPI name>}.}

\processtable{ }
{\begin{tabular}{lcc}\toprule
 Package & PyPI name & Version number\\\midrule
 Nengo & \texttt{nengo} & Development version, commit \texttt{7d2d24145} \\
 Distilled backend & \texttt{nengo\_distilled} & 0.1.0 \\
 OpenCL backend & \texttt{nengo\_ocl} & 0.1.0 \\
 Brainstorm backend & Not available on PyPI & Development version, commit \texttt{ac3cfa708} \\
 SpiNNaker backend & \texttt{nengo\_spinnaker} & 0.2.3 \\
 NumPy & \texttt{numpy} & 1.8.1 \\\botrule
\end{tabular}}{}
\end{table}

\begin{figure}[!ht]
\begin{center}
  \includegraphics[width=85mm]{fig7}
\end{center}
\textbf{\refstepcounter{figure}\label{fig:01} Supplementary Figure \arabic{figure}.}
       {Run time for each model with probes removed on each backend.
         Each bar represents the mean run time across 100 instances
         of each model;
         error bars are 95\% confidence intervals.
         Run time is measured relative to real time;
         e.g., a value of two indicates that the model ran twice as slow
         as real time.}
\end{figure}

\end{document}
