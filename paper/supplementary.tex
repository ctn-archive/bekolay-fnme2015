\documentclass{frontiers_suppmat} % for all articles

\usepackage{url}
\usepackage{inconsolata}
\usepackage{listings}
\lstset{language=Python}
\lstset{basicstyle=\ttfamily,breaklines=true}

\graphicspath{{../figures/}}

\def\articleType{Supplementary Material}
\def\keyFont{\fontsize{8}{11}\helveticabold }
\def\firstAuthorLast{Bekolay {et~al.}}
\def\Authors{Trevor Bekolay\,$^{1,*}$,
  Terrence C. Stewart\,$^{1}$, and
  Chris Eliasmith\,$^1$}
\def\Address{$^{1}$Centre for Theoretical Neuroscience,
  University of Waterloo,
  Waterloo, ON, Canada}
\def\corrAuthor{Trevor Bekolay}
\def\corrAddress{Centre for Theoretical Neuroscience,
  David R. Cheriton School of Computer Science,
  University of Waterloo,
  200 University Avenue West,
  Waterloo, ON  N2L~3G1, Canada}
\def\corrEmail{tbekolay@uwaterloo.ca}

\begin{document}
\onecolumn
\firstpage{1}

\title[Supplementary Material]{{\helveticaitalic{Supplementary Material}}:\\ \helvetica{ Benchmarking neuromorphic systems with Nengo}}
\author[\firstAuthorLast ]{\Authors}
\address{}
\correspondance{}
\extraAuth{}
\topic{Benchmarks and Challenges for Neuromorphic Engineering}

\maketitle

\section{Supplementary Tables and Figures}

\begin{table}[!ht]
\textbf{\refstepcounter{table}\label{Tab:01} Supplementary Table \arabic{table}.}{
  Software versions used for benchmarking.
  The PyPI name is the unique identifier used for installing the Python package
  through the Python Package Index (PyPI). Packages can be found by
  visiting \url{https://pypi.python.org/pypi/<PyPI name>}.}

\processtable{ }
{\begin{tabular}{lcc}\toprule
 Package & PyPI name & Version number\\\midrule
 Nengo & \texttt{nengo} & Development version, commit \texttt{7d2d24145} \\
 Distilled backend & \texttt{nengo\_distilled} & 0.1.0 \\
 OpenCL backend & \texttt{nengo\_ocl} & 0.1.0 \\
 Brainstorm backend & Not available on PyPI & Development version, commit \texttt{ac3cfa708} \\
 SpiNNaker backend & \texttt{nengo\_spinnaker} & 0.2.4 \\
 NumPy & \texttt{numpy} & 1.8.1 \\\botrule
\end{tabular}}{}
\end{table}

\begin{figure}[!ht]
\begin{center}
  \includegraphics[width=180mm]{fig7}
\end{center}
\textbf{\refstepcounter{figure}\label{fig:01} Supplementary Figure \arabic{figure}.}
       {Run time for each model with probes removed on each backend.
         Each bar represents the mean run time across 100 instances
         of each model;
         error bars are 95\% confidence intervals.
         Run time is measured relative to real time;
         e.g., a value of two indicates that the model ran twice as slow
         as real time.
         BG Sequence * refers to the basal ganglia sequence model
         in which passthrough nodes have been removed.}
\end{figure}

\begin{figure}[!ht]
\begin{lstlisting}
import numpy as np
import nengo
from nengo.utils.numpy import rmse

def test_cchannelchain(Simulator, analytics, plt, rng, seed):
    # Parameters that can be varied to investigate extreme cases
    dims = 2
    layers = 5
    n_neurons = 100
    synapse = nengo.Lowpass(0.01)

    with nengo.Network(seed=seed) as model:
        hypersphere = nengo.dists.UniformHypersphere()
        value = hypersphere.sample(dims, 1, rng=rng).ravel()
        stim = nengo.Node(value)

        ens = [nengo.Ensemble(n_neurons, dimensions=dims)
               for _ in range(layers)]

        nengo.Connection(stim, ens[0])
        for i in range(layers - 1):
            nengo.Connection(ens[i], ens[i+1], synapse=synapse)

        p_input = nengo.Probe(stim)
        p_outputs = [nengo.Probe(ens[i], synapse=synapse)
                     for i in range(layers)]

    sim = Simulator(model)
    sim.run(0.5)

    for p_output in p_outputs:
        plt.plot(sim.trange(), sim.data[p_output])
    plt.plot(sim.trange(), sim.data[p_input], color='k', linewidth=1)
    plt.ylabel('Decoded output')
    plt.xlabel('Time (s)')

    last = p_outputs[-1]
    decoding_rmse = rmse(value, sim.data[last][sim.trange() > 0.4]
    analytics.add_data('rmse', decoding_rmse)

def test_compare_cchannelchain(analytics_data, plt):
    rmses = [d['rmse'] for d in analytics_data]
    plt.bar(np.arange(len(rmses)), rmses, align='center')
    plt.ylabel("RMSE")
\end{lstlisting}
\textbf{\refstepcounter{figure}\label{fig:02} Supplementary Figure \arabic{figure}.}
       {Python code for implementing the chained communication channel model.}
\end{figure}

\end{document}
